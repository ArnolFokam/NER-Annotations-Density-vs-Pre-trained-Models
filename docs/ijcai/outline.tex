%%%% ijcai22.tex

\typeout{IJCAI--22 Instructions for Authors}

% These are the instructions for authors for IJCAI-22.

\documentclass{article}
\pdfpagewidth=8.5in
\pdfpageheight=11in
% The file ijcai22.sty is NOT the same as previous years'
\usepackage{ijcai22}

% Use the postscript times font!
\usepackage{times}
\usepackage{soul}
\usepackage{url}
\usepackage[hidelinks]{hyperref}
\usepackage[utf8]{inputenc}
\usepackage[small]{caption}
\usepackage{graphicx}
\usepackage{amsmath}
\usepackage{amsthm}
\usepackage{booktabs}
% \usepackage{algorithm}
\usepackage{algorithmic}
\urlstyle{same}

% the following package is optional:
\usepackage{latexsym}

% See https://www.overleaf.com/learn/latex/theorems_and_proofs
% for a nice explanation of how to define new theorems, but keep
% in mind that the amsthm package is already included in this
% template and that you must *not* alter the styling.
\newtheorem{example}{Example}
\newtheorem{theorem}{Theorem}

% Following comment is from ijcai97-submit.tex:
% The preparation of these files was supported by Schlumberger Palo Alto
% Research, AT\&T Bell Laboratories, and Morgan Kaufmann Publishers.
% Shirley Jowell, of Morgan Kaufmann Publishers, and Peter F.
% Patel-Schneider, of AT\&T Bell Laboratories collaborated on their
% preparation.

% These instructions can be modified and used in other conferences as long
% as credit to the authors and supporting agencies is retained, this notice
% is not changed, and further modification or reuse is not restricted.
% Neither Shirley Jowell nor Peter F. Patel-Schneider can be listed as
% contacts for providing assistance without their prior permission.

% To use for other conferences, change references to files and the
% conference appropriate and use other authors, contacts, publishers, and
% organizations.
% Also change the deadline and address for returning papers and the length and
% page charge instructions.
% Put where the files are available in the appropriate places.

% PDF Info Is REQUIRED.
% Please **do not** include Title and Author information
\pdfinfo{
/TemplateVersion (IJCAI.2022.0)
}

\title{Data Corruption Impact on Named Entity Recognition for Low Resourced Languages}

% Single author syntax

\author{
    Anonymous
    \affiliations
    Anonymous
    \emails
    anonymous
}

% Multiple author syntax (remove the single-author syntax above and the \iffalse ... \fi here)
\iffalse
\author{
First Author$^1$
\and
Second Author$^2$\and
Third Author$^{2,3}$\And
Fourth Author$^4$
\affiliations
$^1$First Affiliation\\
$^2$Second Affiliation\\
$^3$Third Affiliation\\
$^4$Fourth Affiliation
\emails
\{first, second\}@example.com,
third@other.example.com,
fourth@example.com
}
\fi

\begin{document}

\maketitle

\begin{abstract}
\end{abstract}

\section{Introduction}
\label{sec:introduction}
(1 column)
\begin{itemize}
    \item Data acquisition in low resource languages is not th best
    \item Therefore we may get errors on the data
    \item These languages are some times not collected by experienced speakers as experienced speakers will tend bo either verbal and based in rural areas.
    \item This therefore induces the need to train them which can be cost ineffiecient.
    \item This induces the need quantify the impact that such errors has on our models
    \item To perfrom this we choose to design different corruption that reduces the quality of NER datasets
    \item We choose NER due to the simplicity of designing corruption strategies. Also to due the availability of datasets
    \item Indeed, for translation, it is unclear how the quality of data sets can be altered besides altering the target sentence.
    \item How ever there are virtually infinite ways to build target sentences different from source sentences with varying lavel of similarity
    \item However this makes it difficult to design an unbaised metric that quantify the level fo corruption in the data based its semantic property.
    \item This is why we focus on NER.
    \item Because we then look at corruption at a token len. Since NER is a token classification, annotators usually have to classify a span of words from a sentence into different catergories
    \item Therefore a situation where the annotors misclassifies a token is not far from happening.
    \item Therefore we look at the performance of the model and the certainty of its prediction
    \item Explain why look at these two metrics
    \item Then talk about about the contributions of the paper
    \item Breifly give the results obtained
    \item Finally, talk about the outline of the paper
\end{itemize}



\section{Background}
\label{sec:background}
(1 column)
\begin{itemize}
    \item Talk about dataset for low resource language
    \item Methods to construct them (acquisition, annotations, cleaninig)
    \item Talk about these datasets from a point of view
    \item Talk about NER and different methods
\end{itemize}

\section{Approach}
\label{sec:approach}
(2 columns)

\section{Experimental Setup}
\label{sec:experimental_setup}
(1 column)

\subsection{NER Corpora}

\subsection{Models}

\subsection{Evaluation Metric}

\section{Results and Discussions}
\label{sec:experimental_results}
(2 columns)

\section{Conclusion}
\label{sec:conclusion}
(0.5 column)

%% The file named.bst is a bibliography style file for BibTeX 0.99c
\bibliographystyle{named}
\bibliography{ijcai22}

\end{document}

